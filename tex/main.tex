\documentclass[11pt]{article}
\usepackage[utf8]{inputenc}
\usepackage[T1]{fontenc}
\usepackage[english]{babel}
\usepackage{lmodern}
\usepackage{microtype}
\usepackage{csquotes}
\usepackage{amsmath}
\usepackage{hyperref}

\title{Continuous LaTeX Review Assistant (CLaRA)}
\author{CLaRA-Team}
\date{\today}

\begin{document}
\maketitle

\begin{abstract}
  This document is a quick sanity check for PDF generation.
  It contains prose, math, and a table.
\end{abstract}

\section{Introduction}
% clara: ignore-next-line
The goal is a reprodducible toolchain that checks LaTeX documents locally. The suite combines \emph{ChkTeX}, \emph{Vale}, \emph{toolchain}, and optional LLM-assisted hints.
Well, that's definitely fine.

\section{Example}
A short formula: \(f(x) = x^2 + \int_0^1 t\,dt\).

\section{Subdocument} % CLaRA-LLM: a good example
This is a subdocument. This is a good example.


\begin{table}[h]
  \centering
  \begin{tabular}{l r}
    \textbf{Tool} & \textbf{Version} \\
    ChkTeX & 1.7 \\
    Vale & 3.6.0 \\
  \end{tabular}
  \caption{Core tools in the suite.}
\end{table}

\section{Additional Section}
This paragraph is intentionally simple and is meant to show that the document compiles cleanly. It serves as a placeholder for more content.
This is an intentional typos.
\end{document}