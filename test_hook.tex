\documentclass[11pt]{article}
\usepackage[utf8]{inputenc}
\usepackage[T1]{fontenc}
\usepackage[ngerman]{babel}
\begin{document}
% CLaRA-LLM: Suggestion: Entferne den einleitenden Teil "Hier ist ein Tippfehler:" und liste die Fehler direkt auf. | Rationale: Der Hinweis ist redundant; das bloße Auflisten der Fehler ist klarer und prägnanter.
% CLaRA-LLM: Suggestion: Korrigiere "kackte" zu dem intendeden Wort (z. B. "Kapitel" oder das korrekte Verb) oder entferne es, falls es ein Irrtum ist. | Rationale: Ein falsches Wort stört den Lesefluss und untergräbt die Professionalität des Textes.
% CLaRA-LLM: Suggestion: Schreibe "Yale" nur groß, wenn es sich um einen eigenen Namen/Eigenname handelt; sonst verwende Kleinbuchstaben. | Rationale: Großschreibung ist nur für Eigennamen oder am Satzanfang erforderlich; unnötige Großschreibung wirkt fehlerhaft.
% CLaRA-LLM: Suggestion: Trenne die beiden Fehlermeldungen mit einem Semikolon oder einem Zeilenumbruch, um die Struktur zu verdeutlichen. | Rationale: Eine klare Trennung erleichtert das schnelle Erfassen einzelner Fehler.
% CLaRA-LLM: Suggestion: Vermeide das Wort "Und" am Satzanfang in formellen Texten; starte stattdessen mit einem substantivierten Hauptwort. | Rationale: „Und“ ist im schriftlichen Stil oft unnötig und wirkt umgangssprachlich.
Dies ist ein Test für den pre-commit hook.
Hier ist ein Tippfehler: kackte.
Und noch einer: Yale.
\end{document}
