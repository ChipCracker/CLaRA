\documentclass[11pt]{article}
\usepackage[utf8]{inputenc}
\usepackage[T1]{fontenc}
\usepackage[ngerman]{babel}
\usepackage{lmodern}
\usepackage{microtype}
\usepackage{csquotes}
\usepackage{amsmath}
\usepackage{hyperref}

\title{Continuous LaTeX Review Assistant (CLaRA)}
\author{CLaRA-Team}
\date{\today}

\begin{document}
\maketitle

\begin{abstract}
  Dieses Dokument dient als Schnelltest für die PDF-Erstellung.
  Es enthält Fließtext, Mathematik und eine Tabelle.
\end{abstract}

\section{Einführung}
Ziel ist eine reproduzierbare Werkzeugkette, die LaTeX-Dokumente lokal prüft. Die Suite kombiniert \emph{ChkTeX}, \emph{Vale}, \emph{LanguageTool} und optionale LLM-Hinweise.
% clara: ignore-next-line
Na ja das geht schoon.

\section{Beispiel}
Eine kurze Formel: \(f(x) = x^2 + \int_0^1 t\,dt\).

\section{Subdocument} % CLaRA-LLM: a good example
This is a subdocument. This is a good example.


\begin{table}[h]
  \centering
  \begin{tabular}{l r}
    \textbf{Werkzeug} & \textbf{Version} \\
    ChkTeX & 1.7 \\
    Vale & 3.6.0 \\
  \end{tabular}
  \caption{Kernwerkzeuge der Suite.}
\end{table}

\section{Zusätzlicher Abschnitt}
Dieser Absatz ist bewusst einfach gehalten und soll zeigen, dass das Dokument fehlerfrei kompiliert. Er dient als Platzhalter für weitere Inhalte.
Das ist ein absichtlicher Tippfehler. % CLaRA-FIX: Das Wort 'aabsichtlicher' wird korrigiert zu 'absichtlicher', das ist die kürzeste passende Suggestion und bildet den korrekten deutschen Begriff.
\end{document}